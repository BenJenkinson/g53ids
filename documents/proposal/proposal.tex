\documentclass[a4paper]{article}

% Packages
\usepackage{fullpage}
\usepackage[T1]{fontenc}

% Document Title
\title{G53IDS Individual Dissertation Single Honours}
\author{Ben Jenkinson}
\date{October 2011}

% Margins
% \setlength{\textwidth}{5in}
% \setlength{\oddsidemargin}{0.625in}

\begin{document}

	% Import the Title Page
	\begin{titlepage}

	\begin{center}

		\vspace*{3cm}

		\textsc{\LARGE University of Nottingham}\\[0.3cm]
		\textsc{\Large School of Computer Science \& I.T.}\\

		\vspace{1cm}

		\textsc{\Large Individual Dissertation}\\[0.3cm]
		\textsc{\large G53IDS}\\[0.5cm]

		\large \emph{as part of} \textsc{G400}\\[0.3cm]
		\textsc{\Large Computer Science (\textsc{BSc Hons}) }\\[0.3cm]

		\vspace{1cm}

		% Title
		{\huge \textbf{A HTML5 Framework for}}\\[0.3cm]
		{\huge \textbf{Multi-Agent Algorithms}}\\[0.6cm]
		{\huge \textbf{Project Proposal}}\\

		\vspace{1cm}

		% Author
		\emph{by}\\[0.3cm]
		\Large Ben \textsc{Jenkinson}\\[0.3cm]
		\small{\texttt{bxj08u} / \texttt{4082995}}\\[0.5cm]

		\vspace{0.5cm}

		% Date
		{\large \today}\\

		% Fill up the rest of the page.
		\vfill

	\end{center}

\end{titlepage}

	\section{Aims and Objectives}

		\paragraph{The aim of this project shall be to produce a framework to both animate and inspect the progress of a multi-agent algorithm.}

		My intention is to create this framework leveraging \textsc{HTML5}.

		The visualiser client would be distinct from the application or process that is actually running the algorithm simulation, with nothing to connect them but a defined data-format to describe the state of the algorithm simulation. This separation would mean that there is no difference between running the simulation and the client on the same computer, and running the simulation remotely. If the client has been written in HTML5 as a single-page web application, then there is no reason why the client cannot be hosted publicly as well and accessed from anywhere in the world.

		This would open up the possibility of running a complex simulation on a supercomputer, with restricted access, and then providing public access to a visualisation that would be updated in real-time as the simulation progresses. This visualisation could then be designed to be as attractive and intuitive as possible so as to better illustrate the process of the simulation to the user, who may not be of a technical mindset. This arrangement would work quite well for showing a non-technical stakeholder the process and the progress made while the simulation is running over a long time-period.

		Of course there is nothing that says the datastream has to be located on another computer, it could just as easily be a local source for a local client.

		This client could then effectively 'tune-in' to a stream of algorithm state-data output by a completely different machine. This would allow multiple researchers or clients to view the algorithm progress in real time while the intensive computuation is done elsewhere. Independantly, each client could then pause and inspect the data of the visualisation at any time wihtout interfering with any of the other clients.

		All the while, the client could continue downloading the live stream as it is output so that the user can then skip around the timeline of the simulation from whenever they first joined the stream.

	\section{Project Plan}

		\subsection{Tasks}

			\begin{itemize}

				\item Define a protocol, \textsc{API} or data-format for communicating the state of a multi-agent algorithm simulation.

					\begin{itemize}

						\item Research types of multi-agent algorithm.
						\item Research a variety of problems that can be solved by multi-agent algorithms.
						\item Aquire some example problems and example algorithms to use in development of the visualisation client.
						\item Define the protocol/format ensuring it would successfully describe the example problems and algorithms.
						\item Write explanatory documentation for the protocol/format.

					\end{itemize}

					It would be wise to limit the scope of my project to a sub-group of problems that would suit visualisation. Perhaps those areas for which an in-progress inspection of the algorithm is particularly useful.

				\item Investigate the potential for a server-side component for the production of the algorithm data.

					\begin{itemize}

						\item Research existing software for the simulation of multi-agent algorithms.
						\item Identify whether any of the existing software could be adapted to generate the data-format that has been defined.
						\item If it is possible, adapt an existing simulation to provide the required output.

					\end{itemize}

				\item \textit{(Further)} Develop my own component to generate the required output stream of simulation data.

				\item Design the visualisation client, capable of reading a stream of the defined data-format and buffering it for playback through an ideally attractive and intuitive interface.

					\begin{itemize}

						\item Develop the ability to read in the data-stream from a defined location.
						\item Use \textsc{HTML5} offline storage to buffer the data-stream for playback.
						\item Develop a clear and attractive visualisation for the simulation data.
						\item Develop an 'inspector' for the visualisation, whereby playback can be paused and any particular agent can be inspected.
						\item \textit{(Futher)} Draw upon my research of existing visualisation software to determine other useful features for the client to possess, then add them.

					\end{itemize}

			\end{itemize}

		\subsection{Milestones}

			\textit{\small{All dates are rough estimates, plan is subject to change. Official dates are shown in \textbf{bold}}}\\

			\begin{tabular}{r p{11cm}}

				\textbf{2011}\\
				\textbf{28 \textsc{Oct}}						& \textbf{Project proposal \& plan} \\
				4 \textsc{Nov}									& Research - Multi-agent algorithms. \\
				11 \textsc{Nov}									& Development - Definition and documentation of protocol/format. \\
				25 \textsc{Nov}									& Research - HTML5 technologies to read and store the protocol/format into the client. \\
				\textbf{28 \textsc{Nov} - 2 \textsc{Dec}}		& \textbf{Presentation} \\
				16 \textsc{Dec}									& Development - Basic client capable of reading or recieving the protocol/format. \\
				30 \textsc{Dec}									& Development - Client is fully capable of storing the data from the protocol/format and skipping forwards and backwards through time. \\

				\\

				\textbf{2012}\\
				6 \textsc{Jan}									& Research - Use of existing simulation software to generate the protocol/format/feed. \\
				27 \textsc{Jan}									& Development - Design and develop the visualisation of the simulation to be both attractive and intuitive. \\
				10 \textsc{Feb}									& Development - Adapt/develop server-side component to generate the protocol/format from the simulation. \\
				\textbf{24 \textsc{Feb}}						& \textbf{Outline \& first chapter} \\
				30 \textsc{Mar}									& Development - System should be fully working. \\
				\textbf{7 \textsc{May}}							& \textbf{Complete dissertation} \\
				\textbf{16 \textsc{May}}						& \textbf{Demonstration} \\

			\end{tabular}

\end{document}









